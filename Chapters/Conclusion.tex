%Chapter 5

\chapter*{Conclusion} % Main chapter title

Nous avons, durant notre travail, fait une recherche approfondie sur les techniques d'intelligence artificielle utilise par les plus grand laboratoires de recherches dans le monde (Facebook AI Reaserch, Google DeepMind, Montreal Institute for Learning Algorithms et d'autre) en général et applique dans le domaine du traitement d'image plus spécialement. C'est ainsi que nous avons découvert cette nouvelle tendance qui est le Deep Learning vers laquelle pratiquement toutes les recherchent scientifiques qui traitent de grandes masses de donnes essayent d'exporter leur recherche. Ceci étant du a la puissance de ses techniques et aux résultats phénoménaux qu'ils continuent a produire, et qui semblent facilement surpasser les techniques classiques.

De notre part, nous avons tenté une approche dans le problème de recherche d'image par contenue (CBIR). Nous avons essayé, dans ce travail, d'apprendre a la machine a représenter sémantiquement une image, pour pouvoir la comparer avec d'autre.

Différentes techniques furent essayées. Nous avons démontre que cette baisse de precision n’était pas le résultat de la compression de l'espace de représentation (4k => 1k)  mais plus tôt de forcer le réseau a s'exprimer en termes, au lieu de sémantique.

Nous avons aussi montre que l'ajout de certains descripteurs “Handcrafted” n'avait finalement pas autant d'impact sur le résultat final. Ce qui reste logique car il est très peu probable que notre cerveau effectue ce genre d’opérations mathématiques formels, et si nous tentons d'imiter son fonctionnement, alors ces informations peuvent ne pas s’avérer être très utiles.

Une amélioration du temps de recherche peut-être permise grâce a l'apprentissage d'une représentation binaire au lieux d'une représentation par tableau de réels. Cette dernière peut permettre la construction d'un arbre de hachage qui optimisera le temps de recherche et récupération d'image similaires. Une méthode inspire par le travail de [Ale et al. 11] qui ont fait passé leurs images (représentation pixel) dans un autoencoder a base de DBN, et les ont compressé en une représentation binaire. Nous pensons qu'une approche similaire mais qui prend, au lieu des pixels brutes de l'image, l'information sémantique apprise par le réseau a convolution pourrait donner de meilleurs résultats. 


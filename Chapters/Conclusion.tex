%Chapter 5

\chapter{Conclusion} % Main chapter title

Nous avons, durant notre travail, fait une recherche approfondie sur les techniques d'intelligence artificielle utilise par les plus grand laboratoires de recherches dans le monde (Facebook AI Reaserch, Google DeepMind, Montreal Institute for Learning Algorithms et d'autre) en général et applique dans le domaine du traitement d'image plus spécialement. C'est ainsi que nous avons découvert cette nouvelle tendance qui est le Deep Learning vers laquelle pratiquement toutes les recherchent scientifiques qui traitent de grandes masses de donnes essayent d'exporter leur recherche. Ceci étant du a la puissance de ses techniques et aux résultats phénoménaux qu'ils continuent a produire, et qui semblent facilement surpasser les techniques classiques.

De notre part, nous avons tenté une approche dans le problème de recherche d'image par contenue (CBIR). Nous essayons, dans ce travail, d'apprendre a la machine a représenter sémantiquement une image, pour pouvoir la comparer avec d'autre.

Nous avons démontre que cette baisse de precision n’était pas le résultat de la compression de l'espace de représentation (4k => 1k)  mais plus tôt de forcer le réseau a s'exprimer en termes, au lieu de sémantique.

Nous avons aussi montre que l'ajout de certains descripteurs “Handcrafted” n'avait finalement pas autant d'impact sur le résultat final. Ce qui reste logique car il est très peu probable que notre cerveau effectue ce genre d’opérations mathématiques formels, et si nous tentons d'imiter son fonctionnement, alors ces informations peuvent ne pas s’avérer être très utiles.

Une amélioration du temps de recherche peut-être permise grâce a l'apprentissage d'une représentation binaire au lieux d'une représentation par tableau de réels. Cette dernière peut permettre la construction d'un arbre de hachage qui optimisera le temps de recherche et récupération d'image similaires. Une méthode inspire par le travail de [CBIR Deep AutoEncoder] qui ont fait passe leurs images (représentation pixel) dans un autoencoder a base de DBN, et les ont compresse en une . Nous pensons qu'une approche similaire mais qui prend, au lieu des pixels brutes de l'image, l'information sémantique apprise par le réseau a convolution pourrait donner de meilleurs résultats. 


[Shr et al. 13] Meenakshi Shruti Pal, Sushil Kumar Garg,  International Journal of Advanced Research in Computer Engineering and Technology (IJARCET) Volume 2, Issue 6, June 2013

[Oja et al. 02] T. Ojala, M. Pietikinen, T. Menp, Multiresolution Gray-Scale and Rotation In-
variant Texture Classification with Local Binary Patterns. IEEE Transactions on
Pattern Analysis and Machine Intelligence 24 971 987, 2002.

[Har 79] R. Haralick, Statistical ans structural approaches to texture, Proc. Of IEEE, vol.
67, no. 5, pp. 786-804, May 1979.

[Low 99] David G. Lowe, Object recognition from local scale-invariant features , Proceedings of the International Conference on Computer Vision, vol. 2,‎ 1999, p. 1150–1157.

[Sin et al. 15] Singh, Anshuman Vikram, "Content-Based Image Retrieval using Deep Learning" (2015). Thesis. Rochester Institute of Technology.  Accessed from http://scholarworks.rit.edu/theses/88286/2015

[Ma et al. 99] Wei-Ying Ma, B.S. Manjunath, NeTra: A toolbox for navigating large image databases, Journal
Multimedia Systems - Special issue on video content based retrieval archive Volume 7 Issue 3, May 1999 Pages 184 - 198 

[Car et al. 99] Chad Carson, Megan Thomas, Serge Belongie, Joseph M. Hellerstein and Jitendra Malik, Blobworld: a System for Region-Based Image Indexing and Retrieval (long version) EECS Department,University of California, Berkeley,1999


[Yod 08] Josiah Yoder Content-Based Image Retrieval from large Medical Image Databases, Avinash C. Kak, The Purdue Robot Vision Laboratory, School of Electrical and Computer Engineering at Purdue University. 2008

[Nib et al. 93] W. Niblack, R. Barber, W. Equitz, M. D. Flickner, E. H. Glasman, D. Petkovic, P. Yanker, C. Faloutsos, G. Taubin, QBIC project: querying images by content, using color, texture, and shape, in: W. Niblack (Ed.), Storage and Retrieval for Image and Video Databases, Vol. 1908 of SPIE Proceedings,1993, pp. 173-187

[Fli et al.95] Myron Flickner, Harpreet Sawhney, Wayne Niblack, Jonathan Ashley, Qian Huang, Byron Dom, Monika Gorkani, Jim Hafner, Denis Lee, Dragutin Petkovic, David Steele, Peter Yanker, "Query by Image and Video Content: The QBIC System", Computer, vol.28, no. 9, pp. 23-32, September 1995, doi:10.1109/2.410146

[Lux et al. 08] Mathias Lux, Savvas A. Chatzichristofis. LIRE: Lucene Image Retrieval – An Extensible Java CBIR Library. In proceedings of the 16th ACM International Conference on Multimedia, pp. 1085-1088, Vancouver, Canada, 2008

[Lux et al. 13] Mathias Lux, Oge Marques Visual Information Retrieval using Java and LIRE, Morgan Claypool Publishers, 2013

[Des et al. 08] Thomas Deselaers, Daniel Keysers, Hermann Ney Features for Image Retrieval: An Experimental Comparison. Information Retrieval. 2008. Vol. 11. Issue 2. Springer. pp. 77-107.

[Roe et al. 92] Anna W. Roe,a Sarah L. Pallaqb Young H. Kwon, and Mriganka Sur, Visual projections routed to the auditory pathway in ferrets: receptive fields of visual neurons in primary auditory cortex, The Journal of Neuroscience, September 1992, 12(g): 36513664

[Mit et al. 97] Tom Mitchell, McGraw Hill, Machine Learning Textbook 1997.

[Goo et al. 16] Ian Goodfellow, Yoshua Bengio, Aaron Courville, 2016, Deep Learning, Book in preparation for MIT Press, http://www.deeplearningbook.org

[Mar et al. 89] H. Martens and T. Naes, Multivariate Calibration, John Wiley, Sons Inc., New York, 1989.

[Moo 65] Gordon E. Moore, Cramming More Components Onto Integrated Circuits, Electronics, vol. 38,‎ 19 avril 1965

[Ben 14] Yoshua Bengio. Deep Learning. Machine Learning Summer School, Iceland 2014.

[Hub et al. 68] Hubel, D. and Wiesel, T. (1968). Receptive fields and functional architecture of monkey striate cortex. Journal of Physiology (London), 195, 215–243

[LeCun et al. 98] Y. LeCun, L. Bottou, Y. Bengio, and P. Haffner. "Gradient-based learning applied to document recognition." Proceedings of the IEEE, 86(11):2278-2324, November 1998. 

[Kri et al. 12] Krizhevsky, Sutskever, and Hinton, ImageNet Classification with Deep Convolutional Neural Networks, NIPS 2012.

[CS231n 16] CS231n: Convolutional Neural Networks for Visual Recognition (January - March, 2016),Fe-Fei Li, Andrej Karpathy, Justin Johnson, Stanford University.

[Site1] http://deeplearning4j.org/deepautoencoder.html

[Site2] Li Jiang, Musings on Deep Learning https://medium.com/global-silicon-valley/machine-learning-yesterday-today-tomorrow-3d3023c7b519\#.u0qtths18


[Rus et al.,15] Olga Russakovsky*, Jia Deng*, Hao Su, Jonathan Krause, Sanjeev Satheesh, Sean Ma, Zhiheng Huang, Andrej Karpathy, Aditya Khosla, Michael Bernstein, Alexander C. Berg and Li Fei-Fei. (* = equal contribution) ImageNet Large Scale Visual Recognition Challenge. IJCV, 2015. Volume 115, Issue 3 , pp 211-252 

[Li et Wan.,03] Jia Li, James Z. Wang, ``Automatic linguistic indexing of pictures by a statistical modeling approach,'' IEEE Transactions on Pattern Analysis and Machine Intelligence, vol. 25, no. 9, pp. 1075-1088, 2003.

[Wan et al.,01] James Z. Wang, Jia Li, Gio Wiederhold, SIMPLIcity: Semantics-sensitive Integrated Matching for Picture LIbraries, IEEE Trans. on Pattern Analysis and Machine Intelligence, vol 23, no.9, pp. 947-963, 2001.

[Fei et al. 04] L. Fei-Fei, R. Fergus and P. Perona. Learning generative visual models from few training examples: an incremental Bayesian approach tested on 101 object categories. IEEE. CVPR 2004, Workshop on Generative-Model Based Vision. 2004

[Cha et al.14] K Chatfield, K Simonyan, A Vedaldi, A Zisserman - Return of the devil in the details: Delving deep into convolutional nets, British Machine Vision Conference, 2014.

[LeCun et al. 89] LeCun, Y., Boser, B., Denker, J.S., Henderson, D., Howard, R.E., Hubbard, W., Jackel, L.D.: Backpropagation applied to handwritten zip code recognition. Neural
Comput. 1(4), 541–551 (1989)

[Zei et al. 14] M.D. Zeiler, R. Fergus Visualizing and Understanding Convolutional Networks, ECCV 2014 (Honourable Mention for Best Paper Award)

[Gru et at. 06] M. Grubinger, P. Clough, and H. Muller and T. Deselaers, "The IAPR TC-12 Benchmark: A New Evaluation Resource" for Visual Information Systems,” Proc. of the Intl. Workshop OntoImage’2006 Language Resources for CBIR, 2006.

[Theano 16] Theano Development Team, Theano: A {Python} framework for fast computation of mathematical expressions, arXiv e-prints, 2016.

[Bart et al. 15] Bart van Merriënboer, Dzmitry Bahdanau, Vincent Dumoulin, Dmitriy Serdyuk, David Warde-Farley, Jan Chorowski, and Yoshua Bengio, "Blocks and Fuel: Frameworks for deep learning," arXiv preprint arXiv:1506.00619 [cs.LG], 2015.

%Chapter 5

\chapter{Introduction / Conclusion} % Main chapter title

Nous avons, durant notre travail, fait une recherche approfondie sur les techniques d'intelligence artificielle utilise par les plus grand laboratoires de recherches dans le monde (Facebook AI Reaserch, Google DeepMind, Montreal Institute for Learning Algorithms et d'autre) en general et applique dans le domaine du traitement d'image plus specialement. C'est ainsi que nous avons decouvert cette nouvelle tendance qui est le Deep Learning vers laquelle pratiquemenet toutes les recherchent scientifiques qui traitent de grandes masses de donnes essayent d'exporter leur recherche. Ceci etant du a la puissance de ses techniquest et aux resultats fenomenaux qu'ils continuent a produire, et qui semblent facilement surpasser les techniques classiques.

De notre part, nous avons tente une approche dans le probleme de recherche d'image par contenue (CBIR). Nous essayons, dans ce travail, d'apprendre a la machine a representer sementiquement une image, pour pouvoir la comparer avec d'autre.

Nous avons demontre que cette baisse de precision n'etait pas le resultat de la compression de l'espace de representation (4k => 1k)  mais plustot de forcer le reseau a s'exprimer en termes, au lieu de sementique.

Nous avons aussi montre que l'ajout de certaine descripteuer “handcrafted” n'avait finalement pas autant d'impact sur le resultat final. Ce resultlats reste logique car il est tres peu probable que notre cerveau effectue ce genre d'operation mathematiques formels, et si nous tentons d'imiter son fonctionement, alors ces informations peuvent s'averer ne pas etre tres utiles.

Une amelioration du temps de recherche peut-etre permise grace a l'apprentissage d'une representation binaire au lieux d'une representation par tableau de reels. Cette derniere peut permettre la construction d'un arbre de hashage qui optimisera le temps de recherche et récupération d'image similaires. Une methode inspire par le travail de [CBIR Deep AutoEncoder] qui ont fait passe leurs images (representation pixel) dans un autoencoder a base de DBN, et les ont compresse en une . Nous pensons qu'une approche similaire mais qui prend, au lieu des pixels brutes de l'image, l'information sementique apprise par le reseau a convolution pourrait donner de meilleurs resultats. 








[1| Roe et all] Visual projections routed to the auditory pathway in ferrets: receptive fields of visual neurons in primary auditory cortex ; Roe et all 1992. 

[2| Meenakshi, Sushil] Image Retrieval: A Literature Review. IJARCET Volume 2, Issue 6, June 2013

[3| Singh] Singh, Anshuman Vikram, "Content-Based Image Retrieval using Deep Learning" (2015). Thesis. Rochester Institute of Technology.  Accessed from http://scholarworks.rit.edu/theses/88286/2015

[4| Josiah ] Josiah Yoder , Content-Based Image Retrieval from large Medical Image Databases. PURDUE ROBOT VISION LAB 

[5| Isphpreet] Ishpreet Singh Virk, Raman Maini Content Based Image Retrieval: Tools and Techniques.  University College of Engineering, Punjabi University, Patiala.

[6| Niblack et all] W. Niblack, R. Barber, W. Equitz, M. D. Flickner, E. H. Glasman, D. Petkovic, P. Yanker, C. Faloutsos, G. Taubin, QBIC project: querying images by content, using color, texture, and shape, in: W. Niblack (Ed.), Storage and Retrieval for Image and Video Databases, Vol. 1908 of SPIE Proceedings,1993, pp. 173-187








[Roe et al. 92] Anna W. Roe,a Sarah L. Pallaqb Young H. Kwon, and Mriganka Sur, Visual projections routed to the auditory pathway in ferrets: receptive fields of visual neurons in primary auditory cortex, The Journal of Neuroscience, September 1992, 12(g): 36513664

[Mit et al. 97] Tom Mitchell, McGraw Hill, Machine Learning Textbook 1997.

[Goo et al. 16] Ian Goodfellow, Yoshua Bengio, Aaron Courville, 2016, Deep Learning, Book in preparation for MIT Press, http://www.deeplearningbook.org

[Mar et al. 89] H. Martens and T. Naes, Multivariate Calibration, John Wiley, Sons Inc., New York, 1989.

[Moo 65] Gordon E. Moore, Cramming More Components Onto Integrated Circuits, Electronics, vol. 38,‎ 19 avril 1965

[Ben 14] Yoshua Bengio. Deep Learning. Machine Learning Summer School, Iceland 2014.

[Hub et al. 68] Hubel, D. and Wiesel, T. (1968). Receptive fields and functional architecture of monkey striate cortex. Journal of Physiology (London), 195, 215–243

[LeCun et al. 98] Y. LeCun, L. Bottou, Y. Bengio, and P. Haffner. "Gradient-based learning applied to document recognition." Proceedings of the IEEE, 86(11):2278-2324, November 1998. 

[Kri et al. 12] Krizhevsky, Sutskever, and Hinton, ImageNet Classification with Deep Convolutional Neural Networks, NIPS 2012.

[CS231n 16] CS231n: Convolutional Neural Networks for Visual Recognition (January - March, 2016),Fe-Fei Li, Andrej Karpathy, Justin Johnson, Stanford University.

[Site1] http://deeplearning4j.org/deepautoencoder.html

[Site2] Li Jiang, Musings on Deep Learning https://medium.com/global-silicon-valley/machine-learning-yesterday-today-tomorrow-3d3023c7b519\#.u0qtths18



[Rus et al.,15] Olga Russakovsky*, Jia Deng*, Hao Su, Jonathan Krause, Sanjeev Satheesh, Sean Ma, Zhiheng Huang, Andrej Karpathy, Aditya Khosla, Michael Bernstein, Alexander C. Berg and Li Fei-Fei. (* = equal contribution) ImageNet Large Scale Visual Recognition Challenge. IJCV, 2015. Volume 115, Issue 3 , pp 211-252 

[Li et Wan.,03] Jia Li, James Z. Wang, ``Automatic linguistic indexing of pictures by a statistical modeling approach,'' IEEE Transactions on Pattern Analysis and Machine Intelligence, vol. 25, no. 9, pp. 1075-1088, 2003.

[Wan et al.,01] James Z. Wang, Jia Li, Gio Wiederhold, SIMPLIcity: Semantics-sensitive Integrated Matching for Picture LIbraries, IEEE Trans. on Pattern Analysis and Machine Intelligence, vol 23, no.9, pp. 947-963, 2001.

[Fei et al. 04] L. Fei-Fei, R. Fergus and P. Perona. Learning generative visual models from few training examples: an incremental Bayesian approach tested on 101 object categories. IEEE. CVPR 2004, Workshop on Generative-Model Based Vision. 2004

[Cha et al.14] K Chatfield, K Simonyan, A Vedaldi, A Zisserman - Return of the devil in the details: Delving deep into convolutional nets, British Machine Vision Conference, 2014.

%exists above
%[CS231n 16] CS231n: Convolutional Neural Networks for Visual Recognition (January - March, 2016),Fe-Fei Li, Andrej Karpathy, Justin Johnson, Stanford University.



[LeCun et al. 89] LeCun, Y., Boser, B., Denker, J.S., Henderson, D., Howard, R.E., Hubbard, W., Jackel, L.D.: Backpropagation applied to handwritten zip code recognition. Neural
Comput. 1(4), 541–551 (1989)

[Zei et al. 14] M.D. Zeiler, R. Fergus Visualizing and Understanding Convolutional Networks, ECCV 2014 (Honourable Mention for Best Paper Award)

[Gru et at. 06] M. Grubinger, P. Clough, and H. Muller and T. Deselaers, "The IAPR TC-12 Benchmark: A New Evaluation Resource" for Visual Information Systems,” Proc. of the Intl. Workshop OntoImage’2006 Language Resources for CBIR, 2006.

%exists above
%[Goo et al. 16] Ian Goodfellow Yoshua Bengio and Aaron Courville, 2016, Deep Learning, Book in preparation for MIT Press, http://www.deeplearningbook.org

%exists above
%[LeCun et al. 98] Y. LeCun, L. Bottou, Y. Bengio, and P. Haffner. "Gradient-based learning applied to document recognition." Proceedings of the IEEE, 86(11):2278-2324, November 1998. 

[Oja et al. 02] T. Ojala, M. Pietikinen, T. Menp, Multiresolution Gray-Scale and Rotation In-
variant Texture Classification with Local Binary Patterns. IEEE Transactions on
Pattern Analysis and Machine Intelligence 24 971 987, 2002.

[Har 79] R. Haralick, Statistical ans structural approaches to texture, Proc. Of IEEE, vol.
67, no. 5, pp. 786-804, May 1979.

[Des et al. 08] Thomas Deselaers, Daniel Keysers, Hermann Ney Features for Image Retrieval: An Experimental Comparison. Information Retrieval. 2008. Vol. 11. Issue 2. Springer. pp. 77-107.

[Lux et al. 13] Mathias Lux, Oge Marques Visual Information Retrieval using Java and LIRE, Morgan Claypool Publishers, 2013


[Theano 16] Theano Development Team, Theano: A {Python} framework for fast computation of mathematical expressions, arXiv e-prints, 2016.

[Bart et al. 15] Bart van Merriënboer, Dzmitry Bahdanau, Vincent Dumoulin, Dmitriy Serdyuk, David Warde-Farley, Jan Chorowski, and Yoshua Bengio, "Blocks and Fuel: Frameworks for deep learning," arXiv preprint arXiv:1506.00619 [cs.LG], 2015.
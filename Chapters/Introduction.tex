%Chapter 5

\chapter*{Introduction générale} % Main chapter title

	Durant ces dernières années, beaucoup de chercheurs de plusieurs domaines et surtout dans de très grands laboratoires de recherche, se mettent à suivre une nouvelle tendance, connue sous le nom de Deep Learning ou apprentissage profond. Cette technologie d'apprentissage automatique est basée sur les réseaux de neurones artificiels, et elle a complètement bouleversé le domaine de l'intelligence artificielle.
	
	Même en étant très récent, l'apprentissage profond commence à faire partie de notre quotidien, de la reconnaissance faciale, le tagging automatique de morceaux de musique, la reconnaissance vocale, l'étiquetage automatique d’images et même la conception de nouvelles molécules pharmaceutiques, etc.

	Nous visons en premier, dans notre travail, à explorer et à essayer de comprendre ce phénomène. Nous essayerons de comprendre les nouvelles techniques que propose l'apprentissage profond en les appliquant dans un problème très connu en vision artificielle, qui est la recherche d'image par le contenu. Dû aux résultats très prometteurs dans beaucoup de domaines, et surtout dans celui du traitement d'image, qui ont été atteint grâce à l'apprentissage profond. Nous nous attendons à ce que les nôtres soient appréciables.
	
	L’idée de notre approche est de permettre à la machine de trouver une représentation plus significative de son contenu. Cette représentation sera différente des représentations classiques qui utilisent des descripteurs, tel que la texture, la couleur, la forme, etc. et ce, afin de pouvoir effectuer des recherches plus correctes.\\

	En premier lieu, nous explorerons les réseaux à convolutions et leurs potentiel à générer des représentations complexes et multiples des images. Puis nous aborderons un autre concept de l'apprentissage profond qui sont les Deep Autoencoders qui vont nous permettre d'unifier les représentation des images. Enfin, nous introduirons des descripteurs de textures aux représentations pour voir si cela peut apporter des améliorations dans les résultats de recherche.\\

	Comme les méthodes d’apprentissage profond sont des méthodes d'apprentissage automatique. L'une des complications à laquelle nous allons faire fasse est d'essayer de comprendre et d'interpréter les résultats obtenus. Nous chercherons à trouver un support théorique à ce que l'apprentissage (qui a toujours été perçu comme une boite noire) accompli. Nous considérons ceci comme étant l'un des objectifs principaux de notre travail.\\

Nous essayerons, enfin, de présenter quelques perspectives et idées qui pourront faire l'objet d'autres recherches et de permettre l'obtention de meilleur résultats.

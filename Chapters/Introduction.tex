%Chapter 5

\chapter{Introduction} % Main chapter title

Ces dernières années, beaucoup de chercheurs dans plusieurs domaines, et surtout, dans de très grands laboratoires de recherches, se mettent a suivre une nouvelle tendance, celle de l'apprentissage profond (deep learning). Nous visons en premier, dans ce travail, a explorer et essayer de comprendre ce phénomène.\\

Nous allons tenter de comprendre ces nouvelles techniques en les appliquant dans un problème de vision connue: la recherche d'image par contenue. En effet, des résultats très prometteurs dans beaucoup de domaines, surtout dans celui du traitement d'image, ont été atteint  grâce a l'apprentissage profond. Nous nous attendons donc a ce que les nôtres soient appréciables. L’idée est de permettre a la machine de trouver une représentation plus significatif (sémantique) de son contenue. Et ce, afin pouvoir effectuer des recherches plus correctes.\\

Les méthodes d’apprentissage profond sont des méthodes d'apprentissage automatique. Une des complications a laquelle nous allons faire fasse est donc d'essayer de comprendre et interpréter les résultats obtenus. Nous chercherons trouver un soutient théorique a ce que l'apprentissage, qui a toujours été perçu comme une boite noire, accomplis. Nous considérons ceci comme étant un des objectifs principaux de ce travail.\\

Nous essayerons, en fin, de présenter quelques perspectives et idées qui pourront faire l'objet d'autres recherches et de permettre l'obtention de meilleur résultats.
